\documentclass[aspectratio=169,14pt]{beamer}

\usepackage{xcolor}
\usepackage{colortbl}
\usepackage{pgf}
\usepackage{amsmath}
\usepackage{amssymb}
\usepackage{latexsym}
\usepackage{tikz}
\usepackage{pgfplots}
\usepackage{pdfpages}
\usepackage{ulem}
\usepackage{ccicons}

\usetikzlibrary{positioning, shapes}
\usetikzlibrary{decorations.pathmorphing}

\definecolor{shaded}{RGB}{200,0,0}
\usecolortheme[named=shaded]{structure}
\definecolor{stressed}{RGB}{150,40,40}
\setbeamercolor{alerted_text}{fg=stressed}

\setbeamertemplate{navigation symbols}{}
\setbeamersize{text margin left=3mm} 
\setbeamersize{text margin right=3mm} 

\setbeamerfont{frametitle}{size=\huge}

\setbeamertemplate{sidebar right}{default}{}

\makeatletter
\define@key{beamerframe}{nofills}[true]{% top
  \beamer@frametopskip=0pt\relax%
  \beamer@framebottomskip=0pt\relax%
  \beamer@frametopskipautobreak=\beamer@frametopskip\relax%
  \beamer@framebottomskipautobreak=\beamer@framebottomskip\relax%
  \def\beamer@initfirstlineunskip{%
    \def\beamer@firstlineitemizeunskip{%
      \vskip-\partopsep\vskip-\topsep\vskip-\parskip%
      \global\let\beamer@firstlineitemizeunskip=\relax}%
    \everypar{\global\let\beamer@firstlineitemizeunskip=\relax}}
}
\makeatother

\newcommand{\setbackgroundblack}{%
\usebackgroundtemplate{
\begin{pgfpicture}{0in}{0in}{\paperwidth}{\paperheight}
\color{black}
\pgfrect[fill]{\pgfxy(0,-1)}{\pgfpoint{\paperwidth}{2\paperheight}}
\end{pgfpicture}
}
}

\newcommand{\sectionslide}[1]{%
\setbackgroundblack
\begin{frame}[nofills]
\Huge
\vfill
\scaletowidth{\textwidth}{\textcolor{white}{\textbf{#1}}}
\vfill
\end{frame}
\clearbackgroundpicture
}

\definecolor{FootColor}{rgb}{0.322,0.322,0.322}%
\definecolor{FootBackgroundColor}{rgb}{1,1,1}%

\setbeamercolor{bottomcolor}{fg=black,bg=gray!15!white}

%%%%%%%%%%%%%%%%%%%%%%%%%%%%%%%%%%%%%%%%%%%%%%%%%%%%%%%%%%%%%%%%
% stack two things so that they have the same size
\newlength{\firstline}
\newlength{\secondline}
\newcommand{\stacksame}[2]{%
\setlength{\firstline}{\widthof{#1}}%
\setlength{\secondline}{\widthof{#2}}%
\pgfmathsetmacro{\myratio}{\firstline/\secondline}%
\shortstack{#1\\\scalebox{\myratio}{#2}}}

\newlength{\myscalewidth}
\newcommand{\scaletowidth}[2]{%
\setlength{\myscalewidth}{\widthof{#2}}%
\pgfmathsetmacro{\myscaleratio}{#1/\myscalewidth}%
\scalebox{\myscaleratio}{#2}}

\setbeamertemplate{footline}{%
\usebeamerfont{structure}
\footnotesize
\begin{tikzpicture}[overlay,remember picture]%
  \node[opacity=0.8,text opacity=1,anchor=base west,yshift=2pt,xshift=-0.5mm,color=FootColor] (website) at (current page.south west) {https://ximera.cloud/};
%  \node[opacity=0.8,text opacity=1,anchor=base east,yshift=2pt,xshift=0.5mm,color=FootColor,fill=FootBackgroundColor] (twitter) at (current page.south east) {ximera@math.osu.edu};
\end{tikzpicture}
}


%%%%%%%%%%%%%%%%%%%%%%%%%%%%%%%%%%%%%%%%%%%%%%%%%%%%%%%%%%%%%%%%
% I like words in front of faded images

\newcommand{\setbackgroundpicturewhite}[1]{%
\definecolor{FootColor}{rgb}{0.322,0.322,0.322}%
\definecolor{FootBackgroundColor}{rgb}{1,1,1}%
\setbeamercolor{bottomcolor}{fg=black,bg=gray!15!white}%
\usebackgroundtemplate{%
\begin{tikzpicture}[overlay,remember picture]%
\draw[fill=white] (current page.north west) rectangle (current page.south east);%
\node[fill=white,minimum width=\paperwidth,minimum height=\paperheight,yshift=1.5mm] [anchor=north west] (mynode) {\hspace{-1.5mm}\includegraphics[width=\paperwidth]{#1}};%
\end{tikzpicture}%
}}


\newcommand{\settallbackgroundpicturewhite}[1]{%
\definecolor{FootColor}{rgb}{0.322,0.322,0.322}%
\definecolor{FootBackgroundColor}{rgb}{1,1,1}%
\setbeamercolor{bottomcolor}{fg=black,bg=gray!15!white}%
\usebackgroundtemplate{%
\begin{tikzpicture}[overlay,remember picture]%
\draw[fill=white] (current page.north west) rectangle (current page.south east);%
\node[minimum width=\paperwidth,minimum height=\paperheight,yshift=1.5mm] [anchor=north west] (mynode) {\hspace{-1.5mm}\includegraphics[height=\paperheight]{#1}};%
\end{tikzpicture}%
}}


\newcommand{\settallbackgroundpictureblack}[1]{%
\definecolor{FootColor}{rgb}{0.678,0.678,0.678}%
\definecolor{FootBackgroundColor}{rgb}{0,0,0}%
\setbeamercolor{bottomcolor}{fg=black,bg=gray!15!white}%
\usebackgroundtemplate{%
\begin{tikzpicture}[overlay,remember picture]%
\draw[fill=black] (current page.north west) rectangle (current page.south east);%
\node[minimum width=\paperwidth,minimum height=\paperheight,yshift=1.5mm] [anchor=north west] (mynode) {\hspace{-1.5mm}\includegraphics[height=\paperheight]{#1}};%
\end{tikzpicture}%
}}


\newcommand{\setbackgroundpictureblack}[1]{%
\definecolor{FootColor}{rgb}{0.678,0.678,0.678}%
\definecolor{FootBackgroundColor}{rgb}{0,0,0}%
\setbeamercolor{bottomcolor}{fg=white,bg=gray!15!black}%
\usebackgroundtemplate{%
\begin{tikzpicture}[overlay,remember picture]%
\draw[fill=black] (current page.north west) rectangle (current page.south east);%
\node[minimum width=\paperwidth,minimum height=\paperheight,yshift=1.5mm] [anchor=north west] (mynode) {\hspace{-1.5mm}\includegraphics[width=\paperwidth]{#1}};%
\end{tikzpicture}%
}}


\newcommand{\setdarkbackgroundpictureblack}[1]{%
\definecolor{FootColor}{rgb}{0.678,0.678,0.678}%
\definecolor{FootBackgroundColor}{rgb}{0,0,0}%
\setbeamercolor{bottomcolor}{fg=white,bg=gray!15!black}
\usebackgroundtemplate{%
\begin{tikzpicture}[overlay,remember picture]%
\draw[fill=black] (current page.north west) rectangle (current page.south east);%
\node[minimum width=\paperwidth,minimum height=\paperheight,yshift=1.5mm] [anchor=north west] (mynode) {\hspace{-1.5mm}\includegraphics[width=\paperwidth]{#1}};%
\draw[fill=black,opacity=0.75] (current page.north west) rectangle (current page.south east);%
\end{tikzpicture}%
}}%


\newcommand{\setdarkbackgroundpicturewhite}[1]{%
\definecolor{FootColor}{rgb}{0.322,0.322,0.322}%
\definecolor{FootBackgroundColor}{rgb}{1,1,1}%
\setbeamercolor{bottomcolor}{fg=black,bg=gray!15!white}
\usebackgroundtemplate{%
\begin{tikzpicture}[overlay,remember picture]%
\draw[fill=white] (current page.north west) rectangle (current page.south east);%
\node[minimum width=\paperwidth,minimum height=\paperheight,yshift=1.5mm] [anchor=north west] (mynode) {\hspace{-1.5mm}\includegraphics[width=\paperwidth]{#1}};%
\draw[fill=white,opacity=0.75] (current page.north west) rectangle (current page.south east);%
\end{tikzpicture}%
}}%

\newcommand{\settalldarkbackgroundpicturewhite}[1]{%
\definecolor{FootColor}{rgb}{0.322,0.322,0.322}%
\definecolor{FootBackgroundColor}{rgb}{1,1,1}%
\setbeamercolor{bottomcolor}{fg=black,bg=gray!15!white}
\usebackgroundtemplate{%
\begin{tikzpicture}[overlay,remember picture]%
\draw[fill=white] (current page.north west) rectangle (current page.south east);%
\node[minimum width=\paperwidth,minimum height=\paperheight,yshift=1.5mm] [anchor=north west] (mynode) {\hspace{-1.5mm}\includegraphics[height=\paperheight]{#1}};%
\draw[fill=white,opacity=0.75] (current page.north west) rectangle (current page.south east);%
\end{tikzpicture}%
}}%


\newcommand{\settalldarkbackgroundpictureblack}[1]{%
\definecolor{FootColor}{rgb}{0.678,0.678,0.678}%
\definecolor{FootBackgroundColor}{rgb}{0,0,0}%
\setbeamercolor{bottomcolor}{fg=black,bg=gray!15!white}
\usebackgroundtemplate{%
\begin{tikzpicture}[overlay,remember picture]%
\draw[fill=black] (current page.north west) rectangle (current page.south east);%
\node[minimum width=\paperwidth,minimum height=\paperheight,yshift=1.5mm] [anchor=north west] (mynode) {\hspace{-1.5mm}\includegraphics[height=\paperheight]{#1}};%
\draw[fill=black,opacity=0.75] (current page.north west) rectangle (current page.south east);%
\end{tikzpicture}%
}}%

\newcommand{\clearbackgroundpicture}{\usebackgroundtemplate{}%
\definecolor{FootColor}{rgb}{0.322,0.322,0.322}%
\definecolor{FootBackgroundColor}{rgb}{1,1,1}%
\setbeamercolor{bottomcolor}{fg=black,bg=gray!15!white}
}


\begin{document}
\setbeamercolor{background canvas}{bg=white,fg=black}
\usebeamercolor[fg]{background canvas}

%%%%%%%%%%%%%%%%%%%%%%%%%%%%%%%%%%%%%%%%%%%%%%%%%%%%%%%%%%%%%%%%
\clearbackgroundpicture
\begin{frame}[nofills]
  \null\vspace{10pt}  
  \Huge

  \scaletowidth{\textwidth}{\textbf{\TeX\ in the Browser}}
  
  \vfill
  \color{gray}
  \large
      %\scaletowidth{\textwidth}{%}
        \parbox[t]{\widthof{Open Educational Resources}}{Joint Mathematics Meeting \\ Open Educational Resources\\Denver, Colorado\\January 16, 2020}
    \hfill
      %\scaletowidth{\textwidth}{
        \parbox[t]{\widthof{\textbf{Jim Fowler and Bart Snapp}}}{{\phantom{a} \\ \textbf{Jim Fowler and Bart Snapp}} \\
      {The Ohio State University} \\
      {Department of Mathematics}}

  \vspace{24pt}\null
\end{frame}


\begin{frame}
  \Large
  Supported by NSF DUE--1245433 \textcolor{gray}{and} \\
 \phantom{Supported by NSF} DUE--1915294 \textcolor{gray}{and} \\
 \phantom{Supported by NSF} DUE--1915363 \textcolor{gray}{and} \\
 \phantom{Supported by NSF} DUE--1915438 \textcolor{gray}{and} \\
  \quad the Shuttleworth Foundation \textcolor{gray}{and} \\
  \quad OSU Library's textbook affordability grant \textcolor{gray}{and} \\
  \quad Ohio State's Affordable Learning eXchange (ALX).

  \vfill
  \footnotesize Any
  opinions, findings, and conclusions or recommendations expressed in
  this material are those of the author(s) and do not necessarily
  reflect the views of the National Science Foundation.  
\end{frame}

\begin{frame}
  \frametitle{The Problem}
  \Large

  I want to share a \TeX\ file \\
  \quad with my students.

  \vfill
  
  There are questions embedded in the text.

  \vfill
  
  I want to know if my students answer the questions, \\
  \quad and when they do, \\
  I want to reward them with points in the gradebook.
  
\end{frame}

\begin{frame}
  \frametitle{\TeX\ as input for computer algebra}
  \large

  The input \texttt{\$\textbackslash answer\{\textbackslash cos\^{}2 x\}\$} \\
  \quad will create an answer blank, \\
  \quad\quad the correct answer to which is $\cos^2 x$.

  \vfill
  
  \texttt{\textbackslash answer}s can be placed in surprising places, e.g.,
  \[
    \mbox{\texttt{\$\textbackslash sqrt\{\textbackslash frac\{1\}\{\textbackslash answer\{4\}\}\} = \textbackslash frac\{1\}\{2\}\$}}
  \]

  \vfill
  
\end{frame}

%%%%%%%%%%%%%%%%%%%%%%%%%%%%%%%%%%%%%%%%%%%%%%%%%%%%%%%%%%%%%%%%
\begin{frame}
  \frametitle{A Solution}
  \large

  \textbf{Ximera} converts \TeX\ source stored in \texttt{git} repositories, \\
  \quad into both paper worksheets and interactive webpages. \\

  \vfill

  Authors use a build tool called \textbf{xake} \\
  \quad to upload content to a Ximera server.

  \vfill

  Grades are passed back to the LMS via LTI.

  \vfill

  Learning events are logged as xAPI events.
  
  % But uploading content to Ximera is too hard: \\
%  \quad it involves a build tool called \textbf{xake};
%  \quad installations of \TeX\ are so varied.

  \vfill
  
\end{frame}

\begin{frame}
  \begin{center}
    \includegraphics[height=\textheight]{XimeraGraphic.png}
  \end{center}
\end{frame}

\begin{frame}
  \begin{center}
    \includegraphics[height=\textheight]{XimeraGraphicCROSS-OUT.png}
  \end{center}
\end{frame}

\begin{frame}
  \begin{center}
    \includegraphics[height=\textheight]{XimeraGraphicNO-SERVER.png}
  \end{center}
\end{frame}

%%%%%%%%%%%%%%%%%%%%%%%%%%%%%%%%%%%%%%%%%%%%%%%%%%%%%%%%%%%%%%%%
\begin{frame}
  \frametitle{A Solution to the Solution's Problem}

  The \textbf{Ximera Xloud} is a static website, \\
  \quad which instructs the browser to download \texttt{.tex} files from GitHub, \\
  \quad runs \TeX\ in the browser (!) to produce a \texttt{.dvi} file, \\
  \quad which is rendered as HTML. \\

\end{frame}

%%%%%%%%%%%%%%%%%%%%%%%%%%%%%%%%%%%%%%%%%%%%%%%%%%%%%%%%%%%%%%%%
\begin{frame}
  \frametitle{How does this work?}

  I wrote a Pascal compiler, available at\\
  \quad \url{https://github.com/kisonecat/web2js} which\\
  \quad compiles Knuth's (e-)\TeX\ source to WebAssembly

  \vfill
  
  The WebAssembly is instrumented to support asynchronous operations, \\
  \quad e.g., when \texttt{\textbackslash input}ing a file from a website, \\
  \quad\quad \TeX\ is paused while waiting for the fetch.

  \vfill
  
  With \texttt{dvi2html}, the resulting \texttt{.dvi} file can be
  displayed on a webpage.

  \vfill
  
\end{frame}

\begin{frame}
  \frametitle{Why isn't this terribly slow?}
  \Large
  
  Slogan: ``A 1980s computer $\cong$ a web browser.''

  \vfill

  On my machine, \\
  \quad I run \TeX, load lots of packages, then dump core\ldots \\

  \vfill

  In the browser, \\
  \quad resurrect the dead \TeX\ process.
  
  \vfill
  
\end{frame}

\begin{frame}
  \frametitle{What about weird packages?}

  All of \TeX Live is available, \\
  \quad hosted on Amazon S3 under \texttt{https://ximera.cloud/texmf/\ldots}

  \vfill

  When you run \texttt{\textbackslash usepackage\{\ldots \}}, \\
  \quad the page uses a fake (JavaScript) version of \texttt{kpathsea} \\
  \quad to locate the file in the \texttt{texmf} tree, \\
  \quad and download the requested file from \texttt{ximera.cloud}.

  \vfill
  
\end{frame}

\sectionslide{TikZ}

\begin{frame}
  \frametitle{What about TikZ?}
  \Large
  
  Use \texttt{\textbackslash pgfsysdriver\{pgfsys-dvisvgm.def\}} \\
  \quad so we output SVG directly into the DVI stream.

\end{frame}

\sectionslide{TikZJax}

\begin{frame}
  \frametitle{An Application}

  In the \texttt{<head>}, include \\
  \quad\texttt{<link rel="stylesheet" type="text/css" } \\
  \quad\quad\texttt{href="http://tikzjax.com/v1/fonts.css">} \\
  \quad\texttt{<script src="http://tikzjax.com/v1/tikzjax.js"></script>}

  \vfill
  
Then in the \texttt{<body>}, include TikZ code such as \\
\quad\texttt{<script type="text/tikz">} \\
\quad\quad\texttt{\textbackslash begin\{tikzpicture\}} \\
\quad\quad\texttt{\textbackslash draw (0,0) circle (1in);} \\
\quad\quad\texttt{\textbackslash end\{tikzpicture\}} \\
\quad\texttt{</script>}
  
\end{frame}

%%%%%%%%%%%%%%%%%%%%%%%%%%%%%%%%%%%%%%%%%%%%%%%%%%%%%%%%%%%%%%%%
\begin{frame}
  \frametitle{The Future}
  \Large

  With a new \texttt{\textbackslash js} primitive, \\
  \quad interactive TikZ is possible, or \\
  \quad randomly generated exercises.

  \vfill
  
  With \texttt{\textbackslash js}, one can build \texttt{\textbackslash sagemath}.

  \vfill

  There are opportunities for new accessibility features.
  
  \vfill

  Progress and state can be stored with Doenet.
  
\end{frame}

\begin{frame}
  \begin{center}
    \includegraphics[height=\textheight]{Doenet_Logo.png}
  \end{center}
\end{frame}

\begin{frame}
  \frametitle{Record grades from any webpage}

  On your website, load \texttt{doenet.js}, e.g. \\
  \quad\scalebox{0.9}{\texttt{<script src="https://unpkg.com/@doenet/beta/dist/doenet.js"/>}}

  \vfill
  
  Then you can report a grade update via \\
  \quad\texttt{let worksheet = new doenet.Worksheet();} \\
  \quad\texttt{worksheet.progress = 0.17;}

  \vfill

\end{frame}

\clearbackgroundpicture
%%%%%%%%%%%%%%%%%%%%%%%%%%%%%%%%%%%%%%%%%%%%%%%%%%%%%%%%%%%%%%%%
\begin{frame}[nofills]

  \vfill

  \scaletowidth{\textwidth}{\textbf{Thank You}}

  \vfill
  \begin{center}
    \begin{tabular}{rl}
      \textcolor{gray}{Email} & \texttt{ximera@math.osu.edu} \\
      \textcolor{gray}{Website} & \texttt{https://ximera.cloud/} \\[2ex]
      \textcolor{gray}{Read more at} & \texttt{http://go.osu.edu/tugboat} 
    \end{tabular}
  \end{center}
  \vfill
   \vfill
   \null\hfill\ccbysa \\[-2ex]
   \null\hfill\footnotesize{Licensed for reuse under a Creative Commons BY-SA License}

\end{frame}


\end{document}


\begin{frame}
  \frametitle{``Users'' is a too broad category}

  \begin{columns}
    \begin{column}{0.5\textwidth}
      The users of Ximera include
      \begin{itemize}
      \item Authors
      \item Instructors
      \item Education Researchers
      \item Students and learners
      \item Developers and designers
      \end{itemize}
      and their needs are different.      
    \end{column}
    \begin{column}{0.5\textwidth}
      \huge
      A solution must \\
      create workflows \\
      for all users.
    \end{column}
  \end{columns}
  
\end{frame}

%%%%%%%%%%%%%%%%%%%%%%%%%%%%%%%%%%%%%%%%%%%%%%%%%%%%%%%%%%%%%%%%
\begin{frame}[nofills]
  \vfill
  \huge

  \begin{tabular}{@{}l@{ }l@{ }l}
  Easy things & should & be easy. \\
  Hard things & shouldn't & be impossible.
  \end{tabular}
  \vfill
\end{frame}


%%%%%%%%%%%%%%%%%%%%%%%%%%%%%%%%%%%%%%%%%%%%%%%%%%%%%%%%%%%%%%%%

\sectionslide{For Authors\ldots}

\begin{frame}
  \frametitle{Author Workflow}
  \large
  
  \begin{enumerate}
  \item Write \LaTeX\, using \texttt{\textbackslash answer} and the like.
  \item Store your \TeX\ files in a \texttt{git} repository.
  \item Run \texttt{xake bake} to perform \TeX-to-HTML conversion \\
    \quad on any changed files.
  \item Content is published via a \texttt{git push}.
  \end{enumerate}
\end{frame}

\begin{frame}
  \frametitle{\LaTeX\ as an archival format}
  \large

  \LaTeX\ amounts to labeled parentheses  \\
  \quad so it's not so different from XML, \\
  but it has saner semantics for whitespace.

  \vfill

  Authors must be confident that their work will last: \\
  \quad ``Even if everything else fails, \\
  \quad\quad at least \texttt{pdflatex} will run.''

  \vfill
  
  \textbf{Source-level graphics:} TikZ is converted to SVGs.

\end{frame}

\begin{frame}
  \frametitle{Semantic markup}
  \large
  
  \texttt{\textbackslash begin\{multipleChoice\}} \\
  \quad\texttt{\textbackslash choice\{One of them\}} \\
  \quad\texttt{\textbackslash choice[correct]\{Pick me!\}} \\    
  \quad\texttt{\textbackslash choice\{Another one\}} \\
  \quad\texttt{\textbackslash choice\{Not this last one\}} \\
  \texttt{\textbackslash end\{multipleChoice\}}  

\end{frame}

\begin{frame}
  \frametitle{Promote a particular pedagogy}
  \large

  \texttt{\textbackslash begin\{hint\}} \\
  \quad\texttt{What can a hint include?} \\
  \quad\texttt{\textbackslash begin\{multipleChoice\}} \\
  \quad\quad\texttt{\textbackslash choice\{Just static text\}} \\
  \quad\quad\texttt{\textbackslash choice[correct]\{More questions\}} \\
  \quad\texttt{\textbackslash end\{multipleChoice\}}   \\
  \texttt{\textbackslash end\{hint\}}
  
\end{frame}

\begin{frame}
  \frametitle{Promote a particular pedagogy}
  \Large
  
  I want a platform that makes it easy to \\
  \quad present learners with some content \\
  \quad\quad (either through text or video) \\
  \quad and help them check their understanding, \\
  \quad\quad with hints and feedback that \\
  \quad\quad\quad don't just tell the answer, \\
  \quad\quad\quad but rather provide scaffolding questions.
  
\end{frame}

\begin{frame}
  \frametitle{Take version control seriously}
  \large
  
  \textbf{Students see the version of the activity \\
  \quad to which they most recently submitted work.}

  \vfill  
  
  When the author changes a page, \\
  \quad the students receive a real-time notification.

  \vfill  
  
  Everyone sees an ``edit'' button at the top of every page, \\
  \quad so students can contribute to the content.
  
  \vfill

  Researchers can track when a change to an activity \\
  \quad improves student outcomes.

  \vfill

  Immutable resources make caching easier.
  
  \vfill
  \vfill
  \vfill
  
\end{frame}  

\begin{frame}
  \frametitle{Conversion to HTML}
  \large
  
  \texttt{xake bake} is performed in part by \texttt{htlatex} \\
  \quad but relies upon MathJax to render math. \\
  The usage of \texttt{\textbackslash newcommand} is captured \\
  \quad in order to tell MathJax about such \texttt{newcommand}s.

  \vfill

  Conversion to HTML is orthogonal to the \textbf{Ximera server,} \\
  \quad which (among other things) saves student work \\
  \quad\quad and records grades to the LMS gradebook.
  
\end{frame}

\sectionslide{For Instructors\ldots}

\begin{frame}
  \frametitle{Instructor Workflow}
  \large
  
  \begin{enumerate}
  \item Run \texttt{xake lti your.random.key} to generate a secret.
  \item Append \texttt{/lti.xml} to a Ximera URL.
  \item Use that URL as the external tool configuration URL \\
    \quad in your LMS.
  \item Type your key and secret into your LMS.
  \end{enumerate}

\end{frame}

\begin{frame}
  \large
  
  As students complete activities, \\
  \quad a green progress bar fills up, \\
  \quad which also changes the score in the gradebook.

  \vfill
  
  Instructors can watch students working in real-time, \\
  \quad and even chat with them.

  \vfill

  When an instructor changes a student's work, \\
  \quad those changes appear on the student's screen in real-time \\
  \quad\quad ---and vice versa.
\end{frame}

\begin{frame}
  \frametitle{Iterate on design}
  \large
  
  Instructors interpolate between \\
  \quad authors and researchers \\
  since an instructor may make small changes to the content \\
  \quad informed by data on how students have used the content.

  \vfill
  
  A \textbf{statistics} button provides summary statistics to instructors, \\
  \quad facilitating this iteration on design.
\end{frame}

%%%%%%%%%%%%%%%%%%%%%%%%%%%%%%%%%%%%%%%%%%%%%%%%%%%%%%%%%%%%%%%% 
\sectionslide{For Researchers\ldots}

\begin{frame}
  \frametitle{Researcher Workflow}
  \large
  
  \begin{enumerate}
  \item Download new events with \texttt{xake data download}
  \item Then dump all events with \texttt{xake data json}
  \end{enumerate}

  \vfill
  
  Events are stored in the xAPI taxonomy, which amounts to \\
  \quad actor-verb-object triples.

  \vfill
  
  For example, \\
  \quad ``$\langle\mbox{\textit{student}}\rangle$ answered $\langle\mbox{\textit{problem}}\rangle$ correctly.''
  
\end{frame}


\begin{frame}
  \frametitle{Many verbs}
  \large

  \texttt{\textbackslash youtube\{aVTzlhkshTk\}} embeds a YouTube player \\
  \quad for the video with id \texttt{aVTzlhkshTk}.

  \vfill

  Scrubbing through the video generates logs like: \\
  \quad ``While watching $\langle\mbox{\textit{video}}\rangle$, \\
  \quad\quad $\langle\mbox{\textit{student}}\rangle$ skipped from $\langle\mbox{\textit{time A}}\rangle$ to $\langle\mbox{\textit{time B}}\rangle$.'' \\
  %These conform to the xAPI taxonomy.

  A \textbf{map--reduce} task summarizes the event logs for instructors, \\
  \quad so they can see common misconceptions.
  
\end{frame}

%%%%%%%%%%%%%%%%%%%%%%%%%%%%%%%%%%%%%%%%%%%%%%%%%%%%%%%%%%%%%%%%
\sectionslide{For Students\ldots}

\begin{frame}
  \frametitle{Student Workflow}
  \large
  
  \begin{enumerate}
  \item Open the course in the LMS.
  \item Click on a Ximera activity.
  \item Work on the activity, \\
    \quad confident your work is being saved every few seconds.
  \end{enumerate}

  Ximera pages work on mobile devices, \\
  \quad and even partial work is saved.
  
\end{frame}

\begin{frame}
  \frametitle{Live preview}
  \Large
  
  As a student types in his/her answer, \\
  \quad Ximera shows the student \\
  \quad\quad the results of parsing his/her input \\
  \quad in real time.
\end{frame}

%%%%%%%%%%%%%%%%%%%%%%%%%%%%%%%%%%%%%%%%%%%%%%%%%%%%%%%%%%%%%%%%
\sectionslide{For Developers\ldots}

\begin{frame}
  \frametitle{Interactive content}
  \Large

  \vfill
  
  Parallel to \texttt{\textbackslash includegraphics\{picture.png\}},\\
  \quad Ximera has \texttt{\textbackslash includeinteractive\{code.js\}},
  where \texttt{code.js} is a CommonJS or AMD-style module \\
  \quad with a magic dependency \texttt{div} \\
  \quad\quad that provides access to the DOM \\
  \quad and also a dependency \texttt{db}.

  \vfill
  
\end{frame}

\begin{frame}
  \frametitle{The database as a proxied object}
  \Large
  
  Objects on Ximera pages have access to an object \texttt{db} \\
  \quad and changes to \texttt{db} are saved to the server. \\[1ex]
  In fact, \texttt{db} participates in \\
  \quad Fraser's differential synchronization  algorithm, \\
  \quad\quad so multiple browsers can work on the same page.
  
\end{frame}



\begin{frame}
  \frametitle{Etymology of Ximera}

  \large
  Instead of inventing new monsters, \\
  \quad \textbf{Ximera glues together existing monsters.}

  \begin{description}[Problem generation]
    \item[Authentication] GnuPG and LTI
    \item[Interactive content] JavaScript
    \item[Problem generation] Sage\TeX\
    \item[Version control] \texttt{git} for source \textit{and} compiled assets
    \item[Markup]  \LaTeX\ (facilitated by \texttt{htlatex})
    \item[Event Logs] xAPI (also known as ``Tin Can'')
    \end{description}
\end{frame}  

\begin{frame}
  \frametitle{The Future}
  \Large
  \vfill
  
  Refactor Ximera into smaller pieces.

  \vfill
  
  Decentralize content with IPFS.

  \vfill

  Decentralize logging.

  \vfill  
\end{frame}


%%%%%%%%%%%%%%%%%%%%%%%%%%%%%%%%%%%%%%%%%%%%%%%%%%%%%%%%%%%%%%%%
%%%%%%%%%%%%%%%%%%%%%%%%%%%%%%%%%%%%%%%%%%%%%%%%%%%%%%%%%%%%%%%%
%%%%%%%%%%%%%%%%%%%%%%%%%%%%%%%%%%%%%%%%%%%%%%%%%%%%%%%%%%%%%%%%
%%%%%%%%%%%%%%%%%%%%%%%%%%%%%%%%%%%%%%%%%%%%%%%%%%%%%%%%%%%%%%%%
%%%%%%%%%%%%%%%%%%%%%%%%%%%%%%%%%%%%%%%%%%%%%%%%%%%%%%%%%%%%%%%%
%%%%%%%%%%%%%%%%%%%%%%%%%%%%%%%%%%%%%%%%%%%%%%%%%%%%%%%%%%%%%%%%






\begin{frame}
  \frametitle{Getting Started (through CoCalc)}

  In a terminal of a \texttt{https://cocalc.com/} project, run \\
  \quad\texttt{curl -OL http://xandbox.github.io/cocalc/install.sh} \\
  \quad\texttt{source install.sh} \\
  \quad\texttt{cd xandbox} \textcolor{gray}{and look at \texttt{first.tex}.} \\
  \quad\texttt{xake name }$\langle\mbox{\textit{your name}}\rangle$ \\
  \quad\texttt{xake bake} \textcolor{gray}{to convert \LaTeX\ to HTML,} \\
  \quad\texttt{xake frost} \textcolor{gray}{to store the HTML in the git repository,} \\
  \quad\texttt{xake serve} \textcolor{gray}{to push to \texttt{ximera.osu.edu}.} \\
  Find your content at \texttt{https://ximera.osu.edu/}$\langle\mbox{\textit{your name}}\rangle$\texttt{/first}
  
\end{frame}



\begin{frame}
  \frametitle{Taking \TeX\ and Sage\TeX\ seriously}
  \Large
  
\texttt{\textbackslash newcommand\{\textbackslash polynomial\}\{x\^{}2 + x + 1\}\}} \\
\texttt{\$\textbackslash answer\{\textbackslash polynomial\}\$}

  \vfill
  
  \texttt{\$\textbackslash answer\{\textbackslash sage\{derivative(cos(x),x)\}\} = }\\
  \quad\quad\texttt{\textbackslash frac\{d\}\{dx\} \textbackslash cos x\$}

  \vfill
  
\end{frame}

\begin{frame}
  \frametitle{Philosophy of the Problem}
  \Large

  \textbf{Assessment must be open-source.} \\
  \quad To pay Pearson to grade our students \\
  \quad\quad is to outsource our core mission. \\
  \quad The most engaged students will want \\
  \quad\quad to dig into the guts of the assessment platform.

\end{frame}

\begin{frame}
  \frametitle{Philosophy of the Problem}
  \Large
  
  \textbf{Don't re-solve solved problems.} \\
  \quad There are open platforms for randomized problems. \\
  \quad The pain point is around \\
  \quad\quad tracking learner engagement \\
  \quad with activities \\
  \quad\quad that are more conceptual than computational.
  
\end{frame}




\begin{frame}
  \frametitle{Don't invent new protocols.}
  \large
  
  \texttt{ximera.osu.edu} is a GPG keyserver.

  \texttt{ximera.osu.edu} speaks the \texttt{git} protocol.

  \vfill

  So those who have submitted a signed key to \texttt{ximera.osu.edu} \\
  \quad may publish content \textcolor{gray}{(``\texttt{xake serve}'')} \\
  \quad by making a \texttt{git push} of both \TeX\ source and HTML. \\

  \vfill  
  
  The HTML is stored \textcolor{gray}{(by ``\texttt{xake frost}'')} \\
  \quad in commits hanging off the master branch.
\end{frame}



\begin{frame}
  \frametitle{Randomized problem generation}
  \Large

  \vfill

  Produce 1 example with Sage\TeX, say \texttt{problem.tex}

  \vfill

  Create symbolic links to that single example, say \\
  \quad\texttt{ln -s problem.tex problem.ditto1.tex} \\
  \quad\texttt{ln -s problem.tex problem.ditto2.tex} \\
  \quad\texttt{ln -s problem.tex problem.ditto3.tex} \\
  and run \texttt{xake bake}.

  \vfill  
  
\end{frame}

%%% Local Variables:
%%% mode: latex
%%% TeX-master: t
%%% End:
